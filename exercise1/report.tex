\documentclass[]{article}
\usepackage{lmodern}
\usepackage{amssymb,amsmath}
\usepackage{ifxetex,ifluatex}
\usepackage{fixltx2e} % provides \textsubscript
\ifnum 0\ifxetex 1\fi\ifluatex 1\fi=0 % if pdftex
  \usepackage[T1]{fontenc}
  \usepackage[utf8]{inputenc}
\else % if luatex or xelatex
  \ifxetex
    \usepackage{mathspec}
  \else
    \usepackage{fontspec}
  \fi
  \defaultfontfeatures{Ligatures=TeX,Scale=MatchLowercase}
\fi
% use upquote if available, for straight quotes in verbatim environments
\IfFileExists{upquote.sty}{\usepackage{upquote}}{}
% use microtype if available
\IfFileExists{microtype.sty}{%
\usepackage{microtype}
\UseMicrotypeSet[protrusion]{basicmath} % disable protrusion for tt fonts
}{}
\usepackage[margin=1in]{geometry}
\usepackage{hyperref}
\hypersetup{unicode=true,
            pdftitle={Compulsory exercise 1: Group 13},
            pdfauthor={Øyvind Klåpbakken, Martin Outzen Berild and Sindre Henriksen},
            pdfborder={0 0 0},
            breaklinks=true}
\urlstyle{same}  % don't use monospace font for urls
\usepackage{color}
\usepackage{fancyvrb}
\newcommand{\VerbBar}{|}
\newcommand{\VERB}{\Verb[commandchars=\\\{\}]}
\DefineVerbatimEnvironment{Highlighting}{Verbatim}{commandchars=\\\{\}}
% Add ',fontsize=\small' for more characters per line
\usepackage{framed}
\definecolor{shadecolor}{RGB}{248,248,248}
\newenvironment{Shaded}{\begin{snugshade}}{\end{snugshade}}
\newcommand{\KeywordTok}[1]{\textcolor[rgb]{0.13,0.29,0.53}{\textbf{#1}}}
\newcommand{\DataTypeTok}[1]{\textcolor[rgb]{0.13,0.29,0.53}{#1}}
\newcommand{\DecValTok}[1]{\textcolor[rgb]{0.00,0.00,0.81}{#1}}
\newcommand{\BaseNTok}[1]{\textcolor[rgb]{0.00,0.00,0.81}{#1}}
\newcommand{\FloatTok}[1]{\textcolor[rgb]{0.00,0.00,0.81}{#1}}
\newcommand{\ConstantTok}[1]{\textcolor[rgb]{0.00,0.00,0.00}{#1}}
\newcommand{\CharTok}[1]{\textcolor[rgb]{0.31,0.60,0.02}{#1}}
\newcommand{\SpecialCharTok}[1]{\textcolor[rgb]{0.00,0.00,0.00}{#1}}
\newcommand{\StringTok}[1]{\textcolor[rgb]{0.31,0.60,0.02}{#1}}
\newcommand{\VerbatimStringTok}[1]{\textcolor[rgb]{0.31,0.60,0.02}{#1}}
\newcommand{\SpecialStringTok}[1]{\textcolor[rgb]{0.31,0.60,0.02}{#1}}
\newcommand{\ImportTok}[1]{#1}
\newcommand{\CommentTok}[1]{\textcolor[rgb]{0.56,0.35,0.01}{\textit{#1}}}
\newcommand{\DocumentationTok}[1]{\textcolor[rgb]{0.56,0.35,0.01}{\textbf{\textit{#1}}}}
\newcommand{\AnnotationTok}[1]{\textcolor[rgb]{0.56,0.35,0.01}{\textbf{\textit{#1}}}}
\newcommand{\CommentVarTok}[1]{\textcolor[rgb]{0.56,0.35,0.01}{\textbf{\textit{#1}}}}
\newcommand{\OtherTok}[1]{\textcolor[rgb]{0.56,0.35,0.01}{#1}}
\newcommand{\FunctionTok}[1]{\textcolor[rgb]{0.00,0.00,0.00}{#1}}
\newcommand{\VariableTok}[1]{\textcolor[rgb]{0.00,0.00,0.00}{#1}}
\newcommand{\ControlFlowTok}[1]{\textcolor[rgb]{0.13,0.29,0.53}{\textbf{#1}}}
\newcommand{\OperatorTok}[1]{\textcolor[rgb]{0.81,0.36,0.00}{\textbf{#1}}}
\newcommand{\BuiltInTok}[1]{#1}
\newcommand{\ExtensionTok}[1]{#1}
\newcommand{\PreprocessorTok}[1]{\textcolor[rgb]{0.56,0.35,0.01}{\textit{#1}}}
\newcommand{\AttributeTok}[1]{\textcolor[rgb]{0.77,0.63,0.00}{#1}}
\newcommand{\RegionMarkerTok}[1]{#1}
\newcommand{\InformationTok}[1]{\textcolor[rgb]{0.56,0.35,0.01}{\textbf{\textit{#1}}}}
\newcommand{\WarningTok}[1]{\textcolor[rgb]{0.56,0.35,0.01}{\textbf{\textit{#1}}}}
\newcommand{\AlertTok}[1]{\textcolor[rgb]{0.94,0.16,0.16}{#1}}
\newcommand{\ErrorTok}[1]{\textcolor[rgb]{0.64,0.00,0.00}{\textbf{#1}}}
\newcommand{\NormalTok}[1]{#1}
\usepackage{longtable,booktabs}
\usepackage{graphicx,grffile}
\makeatletter
\def\maxwidth{\ifdim\Gin@nat@width>\linewidth\linewidth\else\Gin@nat@width\fi}
\def\maxheight{\ifdim\Gin@nat@height>\textheight\textheight\else\Gin@nat@height\fi}
\makeatother
% Scale images if necessary, so that they will not overflow the page
% margins by default, and it is still possible to overwrite the defaults
% using explicit options in \includegraphics[width, height, ...]{}
\setkeys{Gin}{width=\maxwidth,height=\maxheight,keepaspectratio}
\IfFileExists{parskip.sty}{%
\usepackage{parskip}
}{% else
\setlength{\parindent}{0pt}
\setlength{\parskip}{6pt plus 2pt minus 1pt}
}
\setlength{\emergencystretch}{3em}  % prevent overfull lines
\providecommand{\tightlist}{%
  \setlength{\itemsep}{0pt}\setlength{\parskip}{0pt}}
\setcounter{secnumdepth}{0}
% Redefines (sub)paragraphs to behave more like sections
\ifx\paragraph\undefined\else
\let\oldparagraph\paragraph
\renewcommand{\paragraph}[1]{\oldparagraph{#1}\mbox{}}
\fi
\ifx\subparagraph\undefined\else
\let\oldsubparagraph\subparagraph
\renewcommand{\subparagraph}[1]{\oldsubparagraph{#1}\mbox{}}
\fi

%%% Use protect on footnotes to avoid problems with footnotes in titles
\let\rmarkdownfootnote\footnote%
\def\footnote{\protect\rmarkdownfootnote}

%%% Change title format to be more compact
\usepackage{titling}

% Create subtitle command for use in maketitle
\newcommand{\subtitle}[1]{
  \posttitle{
    \begin{center}\large#1\end{center}
    }
}

\setlength{\droptitle}{-2em}

  \title{Compulsory exercise 1: Group 13}
    \pretitle{\vspace{\droptitle}\centering\huge}
  \posttitle{\par}
  \subtitle{TMA4268 Statistical Learning V2019}
  \author{Øyvind Klåpbakken, Martin Outzen Berild and Sindre Henriksen}
    \preauthor{\centering\large\emph}
  \postauthor{\par}
      \predate{\centering\large\emph}
  \postdate{\par}
    \date{22 februar, 2019}


\begin{document}
\maketitle

\section{Problem 1: Multiple linear
regression}\label{problem-1-multiple-linear-regression}

\begin{Shaded}
\begin{Highlighting}[]
\KeywordTok{library}\NormalTok{(GLMsData)}
\KeywordTok{data}\NormalTok{(}\StringTok{"lungcap"}\NormalTok{)}
\NormalTok{lungcap}\OperatorTok{$}\NormalTok{Htcm=lungcap}\OperatorTok{$}\NormalTok{Ht}\OperatorTok{*}\FloatTok{2.54}
\NormalTok{modelA =}\StringTok{ }\KeywordTok{lm}\NormalTok{(}\KeywordTok{log}\NormalTok{(FEV) }\OperatorTok{~}\StringTok{ }\NormalTok{Age }\OperatorTok{+}\StringTok{ }\NormalTok{Htcm }\OperatorTok{+}\StringTok{ }\NormalTok{Gender }\OperatorTok{+}\StringTok{ }\NormalTok{Smoke, }\DataTypeTok{data=}\NormalTok{lungcap)}
\KeywordTok{summary}\NormalTok{(modelA)}
\end{Highlighting}
\end{Shaded}

\begin{verbatim}
## 
## Call:
## lm(formula = log(FEV) ~ Age + Htcm + Gender + Smoke, data = lungcap)
## 
## Residuals:
##      Min       1Q   Median       3Q      Max 
## -0.63278 -0.08657  0.01146  0.09540  0.40701 
## 
## Coefficients:
##              Estimate Std. Error t value Pr(>|t|)    
## (Intercept) -1.943998   0.078639 -24.721  < 2e-16 ***
## Age          0.023387   0.003348   6.984  7.1e-12 ***
## Htcm         0.016849   0.000661  25.489  < 2e-16 ***
## GenderM      0.029319   0.011719   2.502   0.0126 *  
## Smoke       -0.046067   0.020910  -2.203   0.0279 *  
## ---
## Signif. codes:  0 '***' 0.001 '**' 0.01 '*' 0.05 '.' 0.1 ' ' 1
## 
## Residual standard error: 0.1455 on 649 degrees of freedom
## Multiple R-squared:  0.8106, Adjusted R-squared:  0.8095 
## F-statistic: 694.6 on 4 and 649 DF,  p-value: < 2.2e-16
\end{verbatim}

\textbf{Q1:} The equation for the fitted `modelA' is
\[\log(FEV) = -1.944 + +.0234 \times Age + 0.0168 \times Htcm + 0.0293 \times 
GenderM - 0.0460 \times Smoke \, ,\] where \texttt{GenderM} is 0 if
gender is female and 1 if gender is male. The true model is assumed to
be
\[\log(FEV) = \beta_0 + \beta_1 \times Age + \beta_2 \times Htcm + \beta_3
\times GenderM + \beta_4 \times Smoke + \epsilon \, ,\] where
\(\epsilon \sim \mathcal{N}(0, \sigma^2)\).

\textbf{Q2:}

\begin{itemize}
\tightlist
\item
  \texttt{Estimate} values are the estimates of the coefficients
  \(\hat{\beta}=(X^\intercal X)^{-1}X^\intercal Y\). The intercept
  estimate is the value of \texttt{log(FEV)} if all the covariates are
  0. For the covavariates the \texttt{Estimate} values say how much
  \texttt{log(FEV)} increases with one covariate if all the other
  covariates are kept constant (according to the model). The model point
  estimate of \texttt{log(FEV)} for given covariate values is found by
  using the formula in Q1. So if e.g. \texttt{Age} is increased by 1 and
  all the other covariates are kept constant, the estimate of
  \texttt{log(FEV)} increases by 0.0234.
\item
  \texttt{Std.Error} is the standard error of the estimated coefficients
  (the values under \texttt{Estimate}). This tells how much the
  estimates will vary if we repeat the experiment with the same
  covariate values if our model assumptions are true. The values are the
  diagonal elements of the coefficient estimate covariance matrix
  \(Cov(\hat{\beta})=(X^\intercal X)^{-1}\sigma^2\).
\item
  \texttt{Residual\ standard\ error} is an unbiased estimator for the
  standard deviation \(\sigma\) of the error term \(\epsilon\) in our
  model \(y = \beta^\intercal x +  \epsilon\), where \(\beta\) is the
  vector of coefficients and
  \(x_i = (1, Age_i,  Htcm_i, GenderM_i, Smoke_i)\). It is given by
  \[RSE = \hat{\sigma} = \frac{RSS}{n-p-1} \, , \quad \text{where} \quad RSS = 
    \sum_{i=1}^n(Y_i-\hat{Y})^2\] and n is the number of observations
  and p the number of covariats (4).
\item
  \texttt{F-statistic} is a statistic for the null hypothesis
  \(H_0:\beta_1=\beta_2=\beta_3=\beta_4=0\) vs the alternative
  hypothesis \(H_1\): not all the coefficients (except the intercept)
  are 0. The p-value is very small, so in this case \(H_0\) is clearly
  rejected, i.e.~the regression is significant. The value and the
  p-value is given by
  \[F = \frac{(TSS-RSS)/p}{RSS/(n-p-1)} \sim F_{p, n-p-1}\, , \quad
    TSS = \sum_{i=1}^n(Y_i-\bar{Y})^2 \, .\] \(p\) and \(n-p-1\) are the
  degrees of freedom of the F-distribution.
\end{itemize}

\textbf{Q3:} Multiple R-squared is the proportion of the variability in
the response variable explained by the model, i.e.
\[\frac{TSS-RSS}{TSS} \, .\] So here 81.06\% of the variability is
explained by the model.

\textbf{Q4:}

\begin{Shaded}
\begin{Highlighting}[]
\KeywordTok{library}\NormalTok{(ggplot2)}
\CommentTok{# residuals vs fitted}
\KeywordTok{ggplot}\NormalTok{(modelA, }\KeywordTok{aes}\NormalTok{(.fitted, .stdresid)) }\OperatorTok{+}\StringTok{ }\KeywordTok{geom_point}\NormalTok{(}\DataTypeTok{pch =} \DecValTok{21}\NormalTok{) }\OperatorTok{+}
\StringTok{  }\KeywordTok{geom_hline}\NormalTok{(}\DataTypeTok{yintercept =} \DecValTok{0}\NormalTok{, }\DataTypeTok{linetype =} \StringTok{"dashed"}\NormalTok{) }\OperatorTok{+}
\StringTok{  }\KeywordTok{geom_smooth}\NormalTok{(}\DataTypeTok{se =} \OtherTok{FALSE}\NormalTok{, }\DataTypeTok{col =} \StringTok{"red"}\NormalTok{, }\DataTypeTok{size =} \FloatTok{0.5}\NormalTok{, }\DataTypeTok{method =} \StringTok{"loess"}\NormalTok{) }\OperatorTok{+}
\StringTok{  }\KeywordTok{labs}\NormalTok{(}\DataTypeTok{x =} \StringTok{"Fitted values"}\NormalTok{, }\DataTypeTok{y =} \StringTok{"Standardized residuals"}\NormalTok{,}
       \DataTypeTok{title =} \StringTok{"Fitted values vs. Standardized residuals"}\NormalTok{,}
       \DataTypeTok{subtitle =} \KeywordTok{deparse}\NormalTok{(modelA}\OperatorTok{$}\NormalTok{call))}
\end{Highlighting}
\end{Shaded}

\includegraphics{report_files/figure-latex/unnamed-chunk-2-1.pdf}

\begin{Shaded}
\begin{Highlighting}[]
\CommentTok{# qq-plot of residuals}
\KeywordTok{ggplot}\NormalTok{(modelA, }\KeywordTok{aes}\NormalTok{(}\DataTypeTok{sample =}\NormalTok{ .stdresid)) }\OperatorTok{+}
\StringTok{  }\KeywordTok{stat_qq}\NormalTok{(}\DataTypeTok{pch =} \DecValTok{19}\NormalTok{) }\OperatorTok{+}\StringTok{ }
\StringTok{  }\KeywordTok{geom_abline}\NormalTok{(}\DataTypeTok{intercept =} \DecValTok{0}\NormalTok{, }\DataTypeTok{slope =} \DecValTok{1}\NormalTok{, }\DataTypeTok{linetype =} \StringTok{"dotted"}\NormalTok{) }\OperatorTok{+}
\StringTok{  }\KeywordTok{labs}\NormalTok{(}\DataTypeTok{x =} \StringTok{"Theoretical quantiles"}\NormalTok{, }\DataTypeTok{y =} \StringTok{"Standardized residuals"}\NormalTok{, }
       \DataTypeTok{title =} \StringTok{"Normal Q-Q"}\NormalTok{, }\DataTypeTok{subtitle =} \KeywordTok{deparse}\NormalTok{(modelA}\OperatorTok{$}\NormalTok{call))}
\end{Highlighting}
\end{Shaded}

\includegraphics{report_files/figure-latex/unnamed-chunk-2-2.pdf}

\begin{Shaded}
\begin{Highlighting}[]
\CommentTok{# normality test}
\KeywordTok{library}\NormalTok{(nortest) }
\KeywordTok{ad.test}\NormalTok{(}\KeywordTok{rstudent}\NormalTok{(modelA))}
\end{Highlighting}
\end{Shaded}

\begin{verbatim}
## 
##  Anderson-Darling normality test
## 
## data:  rstudent(modelA)
## A = 1.9256, p-value = 6.486e-05
\end{verbatim}

The plot of standardized residuals shows that there is no clear trend,
i.e.~the residuals seem to be randomly distributed around 0 for all
fitted values. We also can't see any covariance structure from this
plot. So this plot does not indicate that any of our model assumptions
are incorrect.

The normal QQ-plot, however, shows that the standardized residuals are
not perfectly normally distributed. This is confirmed by the low p-value
from the Anderson-Darling normality test, leading to rejection of the
null hypothesis that the residuals are normally distributed.

\textbf{Q5:}

\begin{Shaded}
\begin{Highlighting}[]
\NormalTok{modelB =}\StringTok{ }\KeywordTok{lm}\NormalTok{(FEV }\OperatorTok{~}\StringTok{ }\NormalTok{Age }\OperatorTok{+}\StringTok{ }\NormalTok{Htcm }\OperatorTok{+}\StringTok{ }\NormalTok{Gender }\OperatorTok{+}\StringTok{ }\NormalTok{Smoke, }\DataTypeTok{data=}\NormalTok{lungcap)}
\KeywordTok{summary}\NormalTok{(modelB)}

\CommentTok{# residuals vs fitted}
\KeywordTok{ggplot}\NormalTok{(modelB, }\KeywordTok{aes}\NormalTok{(.fitted, .stdresid)) }\OperatorTok{+}\StringTok{ }\KeywordTok{geom_point}\NormalTok{(}\DataTypeTok{pch =} \DecValTok{21}\NormalTok{) }\OperatorTok{+}
\StringTok{  }\KeywordTok{geom_hline}\NormalTok{(}\DataTypeTok{yintercept =} \DecValTok{0}\NormalTok{, }\DataTypeTok{linetype =} \StringTok{"dashed"}\NormalTok{) }\OperatorTok{+}
\StringTok{  }\KeywordTok{geom_smooth}\NormalTok{(}\DataTypeTok{se =} \OtherTok{FALSE}\NormalTok{, }\DataTypeTok{col =} \StringTok{"red"}\NormalTok{, }\DataTypeTok{size =} \FloatTok{0.5}\NormalTok{, }\DataTypeTok{method =} \StringTok{"loess"}\NormalTok{) }\OperatorTok{+}
\StringTok{  }\KeywordTok{labs}\NormalTok{(}\DataTypeTok{x =} \StringTok{"Fitted values"}\NormalTok{, }\DataTypeTok{y =} \StringTok{"Standardized residuals"}\NormalTok{,}
       \DataTypeTok{title =} \StringTok{"Fitted values vs. Standardized residuals"}\NormalTok{,}
       \DataTypeTok{subtitle =} \KeywordTok{deparse}\NormalTok{(modelB}\OperatorTok{$}\NormalTok{call))}
\end{Highlighting}
\end{Shaded}

\includegraphics{report_files/figure-latex/unnamed-chunk-3-1.pdf}

\begin{Shaded}
\begin{Highlighting}[]
\CommentTok{# qq-plot of residuals}
\KeywordTok{ggplot}\NormalTok{(modelB, }\KeywordTok{aes}\NormalTok{(}\DataTypeTok{sample =}\NormalTok{ .stdresid)) }\OperatorTok{+}
\StringTok{  }\KeywordTok{stat_qq}\NormalTok{(}\DataTypeTok{pch =} \DecValTok{19}\NormalTok{) }\OperatorTok{+}\StringTok{ }
\StringTok{  }\KeywordTok{geom_abline}\NormalTok{(}\DataTypeTok{intercept =} \DecValTok{0}\NormalTok{, }\DataTypeTok{slope =} \DecValTok{1}\NormalTok{, }\DataTypeTok{linetype =} \StringTok{"dotted"}\NormalTok{) }\OperatorTok{+}
\StringTok{  }\KeywordTok{labs}\NormalTok{(}\DataTypeTok{x =} \StringTok{"Theoretical quantiles"}\NormalTok{, }\DataTypeTok{y =} \StringTok{"Standardized residuals"}\NormalTok{, }
       \DataTypeTok{title =} \StringTok{"Normal Q-Q"}\NormalTok{, }\DataTypeTok{subtitle =} \KeywordTok{deparse}\NormalTok{(modelB}\OperatorTok{$}\NormalTok{call))}
\end{Highlighting}
\end{Shaded}

\includegraphics{report_files/figure-latex/unnamed-chunk-3-2.pdf}

\begin{Shaded}
\begin{Highlighting}[]
\CommentTok{# normality test}
\KeywordTok{library}\NormalTok{(nortest) }
\KeywordTok{ad.test}\NormalTok{(}\KeywordTok{rstudent}\NormalTok{(modelB))}
\end{Highlighting}
\end{Shaded}

\begin{verbatim}
## 
## Call:
## lm(formula = FEV ~ Age + Htcm + Gender + Smoke, data = lungcap)
## 
## Residuals:
##      Min       1Q   Median       3Q      Max 
## -1.37656 -0.25033  0.00894  0.25588  1.92047 
## 
## Coefficients:
##              Estimate Std. Error t value Pr(>|t|)    
## (Intercept) -4.456974   0.222839 -20.001  < 2e-16 ***
## Age          0.065509   0.009489   6.904 1.21e-11 ***
## Htcm         0.041023   0.001873  21.901  < 2e-16 ***
## GenderM      0.157103   0.033207   4.731 2.74e-06 ***
## Smoke       -0.087246   0.059254  -1.472    0.141    
## ---
## Signif. codes:  0 '***' 0.001 '**' 0.01 '*' 0.05 '.' 0.1 ' ' 1
## 
## Residual standard error: 0.4122 on 649 degrees of freedom
## Multiple R-squared:  0.7754, Adjusted R-squared:  0.774 
## F-statistic:   560 on 4 and 649 DF,  p-value: < 2.2e-16
## 
## 
##  Anderson-Darling normality test
## 
## data:  rstudent(modelB)
## A = 1.2037, p-value = 0.003853
\end{verbatim}

The model with \texttt{FEV} as response instead of \texttt{log(FEV)} is
fitted an testet in the code above. We see that the proportion of
variability explained is somewhat smaller than before (77.54\%) (we
don't need to look at adjusted R-squared since the number of covariates
in the models are the same). Also, the standardized residuals do not
seem to be distributed around 0 for all fitted values. The studentized
residuals are somewhat more likely to be normally distributed, but still
get a p-value as low as 0.4\% from the Anderson-Darling test. Notable is
the fact that \texttt{Smoke} is no longer considered to have a
significant explanatory power, which makes room for a more parsimonous
model. But then again, smoking probably does affect lung capacity. All
in all \texttt{modelA} is preferable, since our model assumptions seem
more reasonable for \texttt{log(FEV)} as response.

\textbf{Q6:} We wish to test \(H_0: \beta_{Age} = 0\) against
\(H_1: \beta_{Age} \neq 0\). We use the test statistic
\[T_{Age} = \frac{\hat{\beta}_{Age}}{\sqrt{c_{jj}}\hat{\sigma}} \sim t_{n-p-1}\,
,\] where \(c_{jj}\) is the second diagonal element of
\(X^\intercal X)^{-1}\). The calculations are already performed by
\texttt{lm} and printed in the summary before. The t-value is 6.984,
resulting in a p-value of 7.1e-12, so the null hypothesis is rejected at
any reasonable signifiance level, i.e. \texttt{Age} has a significant
effect on the response. Common siginficance levels are 0.05 and 0.01.

\textbf{Q7:}

\begin{Shaded}
\begin{Highlighting}[]
\NormalTok{beta_age =}\StringTok{ }\NormalTok{modelA}\OperatorTok{$}\NormalTok{coefficients[}\DecValTok{2}\NormalTok{][[}\StringTok{"Age"}\NormalTok{]]}
\NormalTok{alpha =}\StringTok{ }\FloatTok{0.01}
\NormalTok{p =}\StringTok{ }\DecValTok{4}
\NormalTok{n =}\StringTok{ }\KeywordTok{nrow}\NormalTok{(lungcap)}
\NormalTok{t =}\StringTok{ }\KeywordTok{qt}\NormalTok{(alpha}\OperatorTok{/}\DecValTok{2}\NormalTok{, n}\OperatorTok{-}\NormalTok{p}\OperatorTok{-}\DecValTok{1}\NormalTok{)}
\NormalTok{se_beta_age =}\StringTok{ }\KeywordTok{sqrt}\NormalTok{(}\KeywordTok{vcov}\NormalTok{(modelA)[}\DecValTok{2}\NormalTok{,}\DecValTok{2}\NormalTok{])}
\NormalTok{CI99 =}\StringTok{ }\KeywordTok{c}\NormalTok{(beta_age }\OperatorTok{+}\StringTok{ }\NormalTok{t}\OperatorTok{*}\NormalTok{se_beta_age, beta_age }\OperatorTok{-}\StringTok{ }\NormalTok{t}\OperatorTok{*}\NormalTok{se_beta_age)}
\NormalTok{CI99}
\end{Highlighting}
\end{Shaded}

\begin{verbatim}
## [1] 0.01473674 0.03203769
\end{verbatim}

A 99\% confidence interval for \(\beta_{Age}\) is found and printed
above, using
\[ P(\hat{\beta}_j-t_{\alpha/2,n-p-1}\sqrt{c_{jj}}\hat{\sigma}
\le \beta_j \le \hat{\beta}_j+t_{\alpha/2,n-p-1}\sqrt{c_{jj}}\hat{\sigma})=
1-\alpha \, ,\] where
\(\sqrt{c_{jj}}\hat{\sigma} = SE(\hat{beta_{Age}})\). This interval
tells us that when repeating the experiment and measuring the response
over and over, 99\% of such intervals will contain the true value
\(\beta_{age}\). Since the interval does not contain the value 0, we can
say for sure that the p-value for the test in Q6 is less than 1\%.

\textbf{Q8:}

\begin{Shaded}
\begin{Highlighting}[]
\NormalTok{new =}\StringTok{ }\KeywordTok{data.frame}\NormalTok{(}\DataTypeTok{Age=}\DecValTok{16}\NormalTok{, }\DataTypeTok{Htcm=}\DecValTok{170}\NormalTok{, }\DataTypeTok{Gender=}\StringTok{"M"}\NormalTok{, }\DataTypeTok{Smoke=}\DecValTok{0}\NormalTok{)}
\NormalTok{log_prediction =}\StringTok{ }\KeywordTok{predict.lm}\NormalTok{(modelA, new, }\DataTypeTok{interval=}\StringTok{"prediction"}\NormalTok{, }\DataTypeTok{level=}\FloatTok{0.95}\NormalTok{)}
\NormalTok{log_prediction}
\NormalTok{prediction =}\StringTok{ }\KeywordTok{exp}\NormalTok{(log_prediction)}
\NormalTok{prediction}
\end{Highlighting}
\end{Shaded}

\begin{verbatim}
##        fit     lwr      upr
## 1 1.323802 1.03616 1.611444
##       fit      lwr      upr
## 1 3.75768 2.818373 5.010038
\end{verbatim}

Our best guess for the value of \texttt{log(FEV)} for new person is
1.324. A 95\% prediction interval for his \texttt{FEV} is (2.818,
5.010). This tells us that we can say with 95\% certainty (given that
our model assumptions are true) that he has a relatively high lung
capacity; possibly close to the sample maximum 5.793 and likely not
close to the minimum 0.791. The interval is, however, quite big.

\section{Problem 2: Classification}\label{problem-2-classification}

\begin{Shaded}
\begin{Highlighting}[]
\KeywordTok{library}\NormalTok{(class)}\CommentTok{# for function knn}
\KeywordTok{library}\NormalTok{(lattice)}
\KeywordTok{library}\NormalTok{(caret)}\CommentTok{# for confusion matrices}
\KeywordTok{library}\NormalTok{(ggplot2)}

\NormalTok{raw =}\StringTok{ }\KeywordTok{read.csv}\NormalTok{(}\StringTok{"https://www.math.ntnu.no/emner/TMA4268/2019v/data/tennis.csv"}\NormalTok{)}
\NormalTok{M =}\StringTok{ }\KeywordTok{na.omit}\NormalTok{(}\KeywordTok{data.frame}\NormalTok{(}\DataTypeTok{y=}\KeywordTok{as.factor}\NormalTok{(raw}\OperatorTok{$}\NormalTok{Result),}
                       \DataTypeTok{x1=}\NormalTok{raw}\OperatorTok{$}\NormalTok{ACE.}\DecValTok{1}\OperatorTok{-}\NormalTok{raw}\OperatorTok{$}\NormalTok{UFE.}\DecValTok{1}\OperatorTok{-}\NormalTok{raw}\OperatorTok{$}\NormalTok{DBF.}\DecValTok{1}\NormalTok{, }
                       \DataTypeTok{x2=}\NormalTok{raw}\OperatorTok{$}\NormalTok{ACE.}\DecValTok{2}\OperatorTok{-}\NormalTok{raw}\OperatorTok{$}\NormalTok{UFE.}\DecValTok{2}\OperatorTok{-}\NormalTok{raw}\OperatorTok{$}\NormalTok{DBF.}\DecValTok{2}\NormalTok{))}
\KeywordTok{set.seed}\NormalTok{(}\DecValTok{4268}\NormalTok{) }\CommentTok{# for reproducibility}
\NormalTok{tr =}\StringTok{ }\KeywordTok{sample.int}\NormalTok{(}\KeywordTok{nrow}\NormalTok{(M),}\KeywordTok{nrow}\NormalTok{(M)}\OperatorTok{/}\DecValTok{2}\NormalTok{)}
\NormalTok{trte=}\KeywordTok{rep}\NormalTok{(}\DecValTok{1}\NormalTok{,}\KeywordTok{nrow}\NormalTok{(M))}
\NormalTok{trte[tr]=}\DecValTok{0}
\NormalTok{Mdf=}\KeywordTok{data.frame}\NormalTok{(M,}\StringTok{"istest"}\NormalTok{=}\KeywordTok{as.factor}\NormalTok{(trte))}
\end{Highlighting}
\end{Shaded}

\textbf{Q9:} The KNN estimator \(\hat{y}\in{0,1}\) is determined by
taking a ``majority vote'' of the N closest neighbours of \(x\).

\begin{equation}
\hat{y} = \hat{P}(Y = j|X = x_0) = \frac{1}{K}\sum\limits_{i\in\mathcal{N}_0}I(y_i = j) = 
\begin{cases}
\frac{1}{K}\sum\limits_{i\in N}y_i, \enspace j = 1\\
1-\frac{1}{K}+\sum\limits_{i\in N} y_i, \enspace j = 0
\end{cases}
\end{equation}

\textbf{Q10:} Finding the the test and training error of tthe data set

\begin{Shaded}
\begin{Highlighting}[]
\NormalTok{ks =}\StringTok{ }\DecValTok{1}\OperatorTok{:}\DecValTok{30}
\NormalTok{train.var =}\StringTok{ }\NormalTok{Mdf[Mdf}\OperatorTok{$}\NormalTok{istest }\OperatorTok{==}\StringTok{ }\DecValTok{0}\NormalTok{, }\DecValTok{2}\OperatorTok{:}\DecValTok{3}\NormalTok{]}
\NormalTok{test.var =}\StringTok{ }\NormalTok{Mdf[Mdf}\OperatorTok{$}\NormalTok{istest }\OperatorTok{==}\StringTok{ }\DecValTok{1}\NormalTok{, }\DecValTok{2}\OperatorTok{:}\DecValTok{3}\NormalTok{]}
\NormalTok{train.def =}\StringTok{ }\NormalTok{Mdf[Mdf}\OperatorTok{$}\NormalTok{istest }\OperatorTok{==}\StringTok{ }\DecValTok{0}\NormalTok{, }\DecValTok{1}\NormalTok{]}
\NormalTok{test.def =}\StringTok{ }\NormalTok{Mdf[Mdf}\OperatorTok{$}\NormalTok{istest }\OperatorTok{==}\StringTok{ }\DecValTok{1}\NormalTok{, }\DecValTok{1}\NormalTok{]}
\NormalTok{train.e =}\StringTok{ }\KeywordTok{numeric}\NormalTok{(}\DecValTok{30}\NormalTok{)}
\NormalTok{test.e =}\StringTok{ }\KeywordTok{numeric}\NormalTok{(}\DecValTok{30}\NormalTok{)}
\ControlFlowTok{for}\NormalTok{ (k }\ControlFlowTok{in}\NormalTok{ ks)\{}
\NormalTok{  model1 <-}\StringTok{ }\NormalTok{class}\OperatorTok{::}\KeywordTok{knn}\NormalTok{(}\DataTypeTok{train =}\NormalTok{ train.var, }\DataTypeTok{test =}\NormalTok{ test.var, }\DataTypeTok{cl =}\NormalTok{ train.def, }\DataTypeTok{k =}\NormalTok{ k, }\DataTypeTok{prob=}\OtherTok{TRUE}\NormalTok{)}
\NormalTok{  model2 <-}\StringTok{ }\NormalTok{class}\OperatorTok{::}\KeywordTok{knn}\NormalTok{(}\DataTypeTok{train =}\NormalTok{ train.var, }\DataTypeTok{test =}\NormalTok{ train.var, }\DataTypeTok{cl =}\NormalTok{ train.def, }\DataTypeTok{k =}\NormalTok{ k, }\DataTypeTok{prob=}\OtherTok{TRUE}\NormalTok{)}
\NormalTok{  train.e[k] =}\StringTok{ }\KeywordTok{mean}\NormalTok{(train.def }\OperatorTok{!=}\StringTok{ }\NormalTok{model2)}
\NormalTok{  test.e[k] =}\StringTok{ }\KeywordTok{mean}\NormalTok{(test.def }\OperatorTok{!=}\StringTok{ }\NormalTok{model1)}
\NormalTok{\}}
\KeywordTok{ggplot}\NormalTok{(}\KeywordTok{data.frame}\NormalTok{(}\DataTypeTok{K=}\DecValTok{1}\OperatorTok{:}\DecValTok{30}\NormalTok{,train.e,test.e),}\KeywordTok{aes}\NormalTok{(}\DataTypeTok{x=}\NormalTok{K)) }\OperatorTok{+}\StringTok{ }
\StringTok{  }\KeywordTok{geom_line}\NormalTok{(}\KeywordTok{aes}\NormalTok{(}\DataTypeTok{y =}\NormalTok{ train.e, }\DataTypeTok{col =} \StringTok{"Train error"}\NormalTok{),}\DataTypeTok{size =} \DecValTok{1}\NormalTok{) }\OperatorTok{+}\StringTok{ }
\StringTok{  }\KeywordTok{geom_line}\NormalTok{(}\KeywordTok{aes}\NormalTok{(}\DataTypeTok{y =}\NormalTok{ test.e, }\DataTypeTok{col =} \StringTok{"Test error"}\NormalTok{),}\DataTypeTok{size =} \DecValTok{1}\NormalTok{) }\OperatorTok{+}\StringTok{ }
\StringTok{  }\KeywordTok{ylab}\NormalTok{(}\StringTok{""}\NormalTok{)}\OperatorTok{+}
\StringTok{  }\KeywordTok{theme}\NormalTok{(}\DataTypeTok{legend.title =} \KeywordTok{element_blank}\NormalTok{())}
\end{Highlighting}
\end{Shaded}

\includegraphics{report_files/figure-latex/unnamed-chunk-7-1.pdf}

In the Figure above the training error and test error is drawn.

\begin{Shaded}
\begin{Highlighting}[]
\KeywordTok{set.seed}\NormalTok{(}\DecValTok{0}\NormalTok{)}
\NormalTok{ks =}\StringTok{ }\DecValTok{1}\OperatorTok{:}\DecValTok{30} \CommentTok{# Choose K from 1 to 30.}
\NormalTok{idx =}\StringTok{ }\KeywordTok{createFolds}\NormalTok{(M[tr,}\DecValTok{1}\NormalTok{], }\DataTypeTok{k=}\DecValTok{5}\NormalTok{) }\CommentTok{# Divide the training data into 5 folds.}
\CommentTok{# "Sapply" is a more efficient for-loop. }
\CommentTok{# We loop over each fold and each value in "ks"}
\CommentTok{# and compute error rates for each combination.}
\CommentTok{# All the error rates are stored in the matrix "cv", }
\CommentTok{# where folds are rows and values of $K$ are columns.}
\NormalTok{cv =}\StringTok{ }\KeywordTok{sapply}\NormalTok{(ks, }\ControlFlowTok{function}\NormalTok{(k)\{ }
  \KeywordTok{sapply}\NormalTok{(}\KeywordTok{seq_along}\NormalTok{(idx), }\ControlFlowTok{function}\NormalTok{(j) \{}
\NormalTok{    yhat =}\StringTok{ }\NormalTok{class}\OperatorTok{::}\KeywordTok{knn}\NormalTok{(}\DataTypeTok{train=}\NormalTok{M[tr[ }\OperatorTok{-}\NormalTok{idx[[j]] ], }\OperatorTok{-}\DecValTok{1}\NormalTok{],}
               \DataTypeTok{cl=}\NormalTok{M[tr[ }\OperatorTok{-}\NormalTok{idx[[j]] ], }\DecValTok{1}\NormalTok{],}
               \DataTypeTok{test=}\NormalTok{M[tr[ idx[[j]] ], }\OperatorTok{-}\DecValTok{1}\NormalTok{], }\DataTypeTok{k =}\NormalTok{ k)}
    \KeywordTok{mean}\NormalTok{(M[tr[ idx[[j]] ], }\DecValTok{1}\NormalTok{] }\OperatorTok{!=}\StringTok{ }\NormalTok{yhat)}
\NormalTok{  \})}
\NormalTok{\})}
\end{Highlighting}
\end{Shaded}

\textbf{Q11:} Calculating the average CV error, the standard error of
the average CV error over the 5 folds and the \(K\) corresponding to the
smallest CV error.

\begin{Shaded}
\begin{Highlighting}[]
\NormalTok{cv.e =}\StringTok{ }\KeywordTok{colMeans}\NormalTok{(cv)}
\NormalTok{cv.se =}\StringTok{ }\KeywordTok{apply}\NormalTok{(cv,}\DecValTok{2}\NormalTok{,sd)}\OperatorTok{/}\KeywordTok{sqrt}\NormalTok{(}\DecValTok{5}\NormalTok{)}
\NormalTok{k.min =}\StringTok{ }\KeywordTok{which.min}\NormalTok{(cv.e)}
\KeywordTok{cat}\NormalTok{(}\StringTok{"k.min = "}\NormalTok{, k.min)}
\end{Highlighting}
\end{Shaded}

\begin{verbatim}
## k.min =  22
\end{verbatim}

\textbf{Q12:}

\begin{Shaded}
\begin{Highlighting}[]
\KeywordTok{library}\NormalTok{(colorspace)}
\NormalTok{co =}\StringTok{ }\KeywordTok{rainbow_hcl}\NormalTok{(}\DecValTok{3}\NormalTok{)}
\KeywordTok{par}\NormalTok{(}\DataTypeTok{mar=}\KeywordTok{c}\NormalTok{(}\DecValTok{4}\NormalTok{,}\DecValTok{4}\NormalTok{,}\DecValTok{1}\NormalTok{,}\DecValTok{1}\NormalTok{)}\OperatorTok{+}\NormalTok{.}\DecValTok{1}\NormalTok{, }\DataTypeTok{mgp =} \KeywordTok{c}\NormalTok{(}\DecValTok{3}\NormalTok{, }\DecValTok{1}\NormalTok{, }\DecValTok{0}\NormalTok{))}
\KeywordTok{plot}\NormalTok{(ks, cv.e, }\DataTypeTok{type=}\StringTok{"o"}\NormalTok{, }\DataTypeTok{pch =} \DecValTok{16}\NormalTok{, }\DataTypeTok{ylim =} \KeywordTok{c}\NormalTok{(}\DecValTok{0}\NormalTok{, }\FloatTok{0.7}\NormalTok{), }\DataTypeTok{col =}\NormalTok{ co[}\DecValTok{2}\NormalTok{],}
     \DataTypeTok{xlab =} \StringTok{"Number of neighbors"}\NormalTok{, }\DataTypeTok{ylab=}\StringTok{"Misclassification error"}\NormalTok{)}
\KeywordTok{arrows}\NormalTok{(ks, cv.e}\OperatorTok{-}\NormalTok{cv.se, ks, cv.e}\OperatorTok{+}\NormalTok{cv.se, }\DataTypeTok{angle=}\DecValTok{90}\NormalTok{, }\DataTypeTok{length=}\NormalTok{.}\DecValTok{03}\NormalTok{, }\DataTypeTok{code=}\DecValTok{3}\NormalTok{, }\DataTypeTok{col=}\NormalTok{co[}\DecValTok{2}\NormalTok{])}
\KeywordTok{lines}\NormalTok{(ks, train.e, }\DataTypeTok{type=}\StringTok{"o"}\NormalTok{, }\DataTypeTok{pch =} \DecValTok{16}\NormalTok{, }\DataTypeTok{ylim =} \KeywordTok{c}\NormalTok{(}\FloatTok{0.5}\NormalTok{, }\FloatTok{0.7}\NormalTok{), }\DataTypeTok{col =}\NormalTok{ co[}\DecValTok{3}\NormalTok{])}
\KeywordTok{lines}\NormalTok{(ks, test.e, }\DataTypeTok{type=}\StringTok{"o"}\NormalTok{, }\DataTypeTok{pch =} \DecValTok{16}\NormalTok{, }\DataTypeTok{ylim =} \KeywordTok{c}\NormalTok{(}\FloatTok{0.5}\NormalTok{, }\FloatTok{0.7}\NormalTok{), }\DataTypeTok{col =}\NormalTok{ co[}\DecValTok{1}\NormalTok{])}
\KeywordTok{legend}\NormalTok{(}\StringTok{"topright"}\NormalTok{, }\DataTypeTok{legend =} \KeywordTok{c}\NormalTok{(}\StringTok{"Test"}\NormalTok{, }\StringTok{"5-fold CV"}\NormalTok{, }\StringTok{"Training"}\NormalTok{), }\DataTypeTok{lty =} \DecValTok{1}\NormalTok{, }\DataTypeTok{col=}\NormalTok{co)}
\end{Highlighting}
\end{Shaded}

\includegraphics{report_files/figure-latex/unnamed-chunk-10-1.pdf}

The bias for \(\hat{y}(x)\) will increase with increasing \(K\). If
\(K\) is small there want be alot of bias, because the predictors are
close to the point we are trying to estimate. The variance on the other
hand will decrease with increasing K. If K is small the random noise
will affect it the estimate more than if there were more points. You can
also see this from the formula of variance.

\textbf{Q13:} The strategy used in the first line is the
\textit{one standard error rule}, which means that we choose the highest
K for which or
\[K = \max\limits_K\{CV(K) \leq CV(\hat{K}) + SE(\hat{K})\},\] where
\(\hat{K}\) is the k.min from Q11.

\begin{Shaded}
\begin{Highlighting}[]
\NormalTok{k =}\StringTok{ }\KeywordTok{tail}\NormalTok{(}\KeywordTok{which}\NormalTok{(cv.e }\OperatorTok{<}\StringTok{ }\NormalTok{cv.e[k.min] }\OperatorTok{+}\StringTok{ }\NormalTok{cv.se[k.min]), }\DecValTok{1}\NormalTok{)}
\NormalTok{size =}\StringTok{ }\DecValTok{100}
\NormalTok{xnew =}\StringTok{ }\KeywordTok{apply}\NormalTok{(M[tr,}\OperatorTok{-}\DecValTok{1}\NormalTok{], }\DecValTok{2}\NormalTok{, }\ControlFlowTok{function}\NormalTok{(X) }\KeywordTok{seq}\NormalTok{(}\KeywordTok{min}\NormalTok{(X), }\KeywordTok{max}\NormalTok{(X), }\DataTypeTok{length.out=}\NormalTok{size))}
\NormalTok{grid =}\StringTok{ }\KeywordTok{expand.grid}\NormalTok{(xnew[,}\DecValTok{1}\NormalTok{], xnew[,}\DecValTok{2}\NormalTok{])}
\NormalTok{grid.yhat =}\StringTok{ }\KeywordTok{knn}\NormalTok{(M[tr,}\OperatorTok{-}\DecValTok{1}\NormalTok{], M[tr,}\DecValTok{1}\NormalTok{], }\DataTypeTok{k=}\NormalTok{k, }\DataTypeTok{test=}\NormalTok{grid)}
\NormalTok{np =}\StringTok{ }\DecValTok{300}
\KeywordTok{par}\NormalTok{(}\DataTypeTok{mar=}\KeywordTok{rep}\NormalTok{(}\DecValTok{2}\NormalTok{,}\DecValTok{4}\NormalTok{), }\DataTypeTok{mgp =} \KeywordTok{c}\NormalTok{(}\DecValTok{1}\NormalTok{, }\DecValTok{1}\NormalTok{, }\DecValTok{0}\NormalTok{))}
\KeywordTok{contour}\NormalTok{(xnew[,}\DecValTok{1}\NormalTok{], xnew[,}\DecValTok{2}\NormalTok{], }\DataTypeTok{z =} \KeywordTok{matrix}\NormalTok{(grid.yhat, size), }\DataTypeTok{levels=}\NormalTok{.}\DecValTok{5}\NormalTok{, }
        \DataTypeTok{xlab=}\KeywordTok{expression}\NormalTok{(}\StringTok{"x"}\NormalTok{[}\DecValTok{1}\NormalTok{]), }\DataTypeTok{ylab=}\KeywordTok{expression}\NormalTok{(}\StringTok{"x"}\NormalTok{[}\DecValTok{2}\NormalTok{]), }\DataTypeTok{axes=}\OtherTok{FALSE}\NormalTok{,}
        \DataTypeTok{main =} \KeywordTok{paste0}\NormalTok{(k,}\StringTok{"-nearest neighbors"}\NormalTok{), }\DataTypeTok{cex=}\FloatTok{1.2}\NormalTok{, }\DataTypeTok{labels=}\StringTok{""}\NormalTok{)}
\KeywordTok{points}\NormalTok{(grid, }\DataTypeTok{pch=}\StringTok{"."}\NormalTok{, }\DataTypeTok{cex=}\DecValTok{1}\NormalTok{, }\DataTypeTok{col=}\NormalTok{grid.yhat)}
\KeywordTok{points}\NormalTok{(M[}\DecValTok{1}\OperatorTok{:}\NormalTok{np,}\OperatorTok{-}\DecValTok{1}\NormalTok{], }\DataTypeTok{col=}\KeywordTok{factor}\NormalTok{(M[}\DecValTok{1}\OperatorTok{:}\NormalTok{np,}\DecValTok{1}\NormalTok{]), }\DataTypeTok{pch =} \DecValTok{1}\NormalTok{, }\DataTypeTok{lwd =} \FloatTok{1.5}\NormalTok{)}
\KeywordTok{legend}\NormalTok{(}\StringTok{"topleft"}\NormalTok{, }\KeywordTok{c}\NormalTok{(}\StringTok{"Player 1 wins"}\NormalTok{, }\StringTok{"Player 2 wins"}\NormalTok{), }
       \DataTypeTok{col=}\KeywordTok{c}\NormalTok{(}\StringTok{"red"}\NormalTok{, }\StringTok{"black"}\NormalTok{), }\DataTypeTok{pch=}\DecValTok{1}\NormalTok{)}
\KeywordTok{box}\NormalTok{()}
\end{Highlighting}
\end{Shaded}

\textbf{Q14:} The optimal choice is \(K = 30\) from Q13.

\begin{Shaded}
\begin{Highlighting}[]
\NormalTok{K=}\KeywordTok{tail}\NormalTok{(}\KeywordTok{which}\NormalTok{(cv.e }\OperatorTok{<}\StringTok{ }\NormalTok{cv.e[k.min] }\OperatorTok{+}\StringTok{ }\NormalTok{cv.se[k.min]), }\DecValTok{1}\NormalTok{)}
  
\NormalTok{KNNclass=class}\OperatorTok{::}\KeywordTok{knn}\NormalTok{(}\DataTypeTok{train=}\NormalTok{M[tr,}\OperatorTok{-}\DecValTok{1}\NormalTok{], }\DataTypeTok{cl=}\NormalTok{M[tr,}\DecValTok{1}\NormalTok{], }\DataTypeTok{test=}\NormalTok{M[}\OperatorTok{-}\NormalTok{tr,}\OperatorTok{-}\DecValTok{1}\NormalTok{], }\DataTypeTok{k =}\NormalTok{ K,}\DataTypeTok{prob=}\OtherTok{TRUE}\NormalTok{)}
\NormalTok{KNNprobwinning=}\KeywordTok{attributes}\NormalTok{(KNNclass)}\OperatorTok{$}\NormalTok{prob}
\NormalTok{KNNprob=}\StringTok{ }\KeywordTok{ifelse}\NormalTok{(KNNclass }\OperatorTok{==}\StringTok{ "0"}\NormalTok{, }\DecValTok{1}\OperatorTok{-}\NormalTok{KNNprobwinning, KNNprobwinning)}

\KeywordTok{library}\NormalTok{(pROC)}
\NormalTok{myRoc <-}\StringTok{ }\KeywordTok{roc}\NormalTok{(}\DataTypeTok{response =}\NormalTok{ M[}\OperatorTok{-}\NormalTok{tr,}\DecValTok{1}\NormalTok{], }\DataTypeTok{predictor =}\NormalTok{ KNNprob, }\DataTypeTok{auc =} \OtherTok{TRUE}\NormalTok{, }\DataTypeTok{ci =} \OtherTok{TRUE}\NormalTok{)}
\NormalTok{interval =}\StringTok{ }\FloatTok{0.25}
\NormalTok{breaks =}\StringTok{ }\KeywordTok{seq}\NormalTok{(}\DecValTok{0}\NormalTok{,}\DecValTok{1}\NormalTok{,interval)}
\KeywordTok{ggplot}\NormalTok{(}\OtherTok{NULL}\NormalTok{, }\KeywordTok{aes}\NormalTok{(}\DataTypeTok{x =}\KeywordTok{rev}\NormalTok{(myRoc}\OperatorTok{$}\NormalTok{specificities), }\DataTypeTok{y =}\KeywordTok{rev}\NormalTok{(myRoc}\OperatorTok{$}\NormalTok{sensitivities)))}\OperatorTok{+}
\StringTok{  }\KeywordTok{geom_segment}\NormalTok{(}\KeywordTok{aes}\NormalTok{(}\DataTypeTok{x =} \DecValTok{0}\NormalTok{, }\DataTypeTok{y =} \DecValTok{1}\NormalTok{, }\DataTypeTok{xend =} \DecValTok{1}\NormalTok{,}\DataTypeTok{yend =} \DecValTok{0}\NormalTok{), }\DataTypeTok{linetype =} \StringTok{"dashed"}\NormalTok{, }\DataTypeTok{alpha =} \FloatTok{0.5}\NormalTok{) }\OperatorTok{+}\StringTok{ }
\StringTok{  }\KeywordTok{geom_step}\NormalTok{() }\OperatorTok{+}
\StringTok{  }\KeywordTok{scale_x_reverse}\NormalTok{(}\DataTypeTok{name =} \StringTok{"Specificity"}\NormalTok{,}\DataTypeTok{limits =} \KeywordTok{c}\NormalTok{(}\DecValTok{1}\NormalTok{,}\DecValTok{0}\NormalTok{), }\DataTypeTok{breaks =}\NormalTok{ breaks, }\DataTypeTok{expand =} \KeywordTok{c}\NormalTok{(}\FloatTok{0.001}\NormalTok{,}\FloatTok{0.001}\NormalTok{)) }\OperatorTok{+}\StringTok{ }
\StringTok{  }\KeywordTok{scale_y_continuous}\NormalTok{(}\DataTypeTok{name =} \StringTok{"Sensitivity"}\NormalTok{, }\DataTypeTok{limits =} \KeywordTok{c}\NormalTok{(}\DecValTok{0}\NormalTok{,}\DecValTok{1}\NormalTok{), }\DataTypeTok{breaks =}\NormalTok{ breaks, }\DataTypeTok{expand =} \KeywordTok{c}\NormalTok{(}\FloatTok{0.001}\NormalTok{, }\FloatTok{0.001}\NormalTok{)) }\OperatorTok{+}\StringTok{ }
\StringTok{  }\KeywordTok{theme}\NormalTok{(}\DataTypeTok{axis.ticks =} \KeywordTok{element_line}\NormalTok{(}\DataTypeTok{color =} \StringTok{"grey80"}\NormalTok{)) }\OperatorTok{+}
\StringTok{  }\KeywordTok{coord_equal}\NormalTok{() }\OperatorTok{+}\StringTok{ }
\StringTok{  }\KeywordTok{annotate}\NormalTok{(}\StringTok{"text"}\NormalTok{, }\DataTypeTok{x =}\NormalTok{ interval}\OperatorTok{/}\DecValTok{2}\NormalTok{, }\DataTypeTok{y =}\NormalTok{ interval}\OperatorTok{/}\DecValTok{2}\NormalTok{, }\DataTypeTok{vjust =} \DecValTok{0}\NormalTok{, }\DataTypeTok{label =} \KeywordTok{paste}\NormalTok{(}\StringTok{"AUC ="}\NormalTok{,}\KeywordTok{sprintf}\NormalTok{(}\StringTok{"%.3f"}\NormalTok{,myRoc}\OperatorTok{$}\NormalTok{auc)))}\OperatorTok{+}
\StringTok{  }\KeywordTok{ggtitle}\NormalTok{(}\StringTok{"ROC 30-nearest neighbors"}\NormalTok{)}
\end{Highlighting}
\end{Shaded}

\includegraphics{report_files/figure-latex/unnamed-chunk-12-1.pdf}

The ROC curve displays the relationship between the percentage of
correctly predicted negatives of true negatives, \emph{specificity}, and
the percentage of correcly predicted positives of true positives,
\emph{sensitivity}. Their respective equations are \[
\textrm{specificity} =  \frac{\textrm{#true negatives}}{\textrm{#negatives}} = \frac{TN}{N},
\] \[
\textrm{sensitivity} =  \frac{\textrm{#true positives}}{\textrm{#positives}} = \frac{TP}{P}.
\] The \(x\)-axis being the specificity is flipped so that specificity
\(= 1\) is to the left in the ROC. The best scenario is if we don't have
any missclassifications, then the ROC curve would be in the top left
corner, and the \emph{``area under the curve''}(AUC) would be 1. We had
AUC \(=\) 0.8178093.

If we have a predictor that classify a fraction \(p\) as positive, then
\(p\) percent of the predictions would be true positives. If we have a
predictor that classify a fraction \(p\) as negative, the \(p\) percent
of the predictions would be true negative. Therefore true positive rate
is equal to true negative rate, and we have a line from (1,0) to (0,1)
as the ROC curve. This would yield a \(\textrm{AUC} = 0.5\).

\textbf{Q15:}

Now we use a different predictor,
\(\tilde{y} = \textrm{argmax}_k (x_k)\)

\begin{Shaded}
\begin{Highlighting}[]
\NormalTok{  quality_score <-}\StringTok{ }\KeywordTok{data.frame}\NormalTok{(Mdf, }\DataTypeTok{y_hat =} \KeywordTok{as.integer}\NormalTok{(M}\OperatorTok{$}\NormalTok{x1}\OperatorTok{>}\NormalTok{M}\OperatorTok{$}\NormalTok{x2))}
\NormalTok{  whoWins <-}\StringTok{ }\KeywordTok{data.frame}\NormalTok{(}
    \DataTypeTok{x1=}\KeywordTok{c}\NormalTok{(}\OperatorTok{-}\DecValTok{100}\NormalTok{,}\DecValTok{20}\NormalTok{,}\DecValTok{20}\NormalTok{,}\OperatorTok{-}\DecValTok{100}\NormalTok{,}\DecValTok{20}\NormalTok{,}\OperatorTok{-}\DecValTok{100}\NormalTok{),}
    \DataTypeTok{x2=}\KeywordTok{c}\NormalTok{(}\OperatorTok{-}\DecValTok{100}\NormalTok{,}\OperatorTok{-}\DecValTok{100}\NormalTok{,}\DecValTok{20}\NormalTok{,}\OperatorTok{-}\DecValTok{100}\NormalTok{,}\DecValTok{20}\NormalTok{,}\DecValTok{20}\NormalTok{),}
    \DataTypeTok{win=}\KeywordTok{c}\NormalTok{(}\DecValTok{1}\NormalTok{,}\DecValTok{1}\NormalTok{,}\DecValTok{1}\NormalTok{,}\DecValTok{2}\NormalTok{,}\DecValTok{2}\NormalTok{,}\DecValTok{2}\NormalTok{)}
\NormalTok{  )}
\KeywordTok{ggplot}\NormalTok{(quality_score)}\OperatorTok{+}
\StringTok{    }\KeywordTok{geom_polygon}\NormalTok{(}\DataTypeTok{data =}\NormalTok{ whoWins, }\KeywordTok{aes}\NormalTok{(}\DataTypeTok{x=}\NormalTok{x1,}\DataTypeTok{y=}\NormalTok{x2,}\DataTypeTok{group=}\NormalTok{win,}\DataTypeTok{fill=}\KeywordTok{factor}\NormalTok{(win)),}\DataTypeTok{alpha=}\FloatTok{0.2}\NormalTok{)}\OperatorTok{+}
\StringTok{    }\KeywordTok{scale_fill_manual}\NormalTok{(}\DataTypeTok{name =} \StringTok{"Decision"}\NormalTok{,}\DataTypeTok{labels=}\KeywordTok{c}\NormalTok{(}\StringTok{"Player 2 wins"}\NormalTok{,}\StringTok{"Player 1 wins"}\NormalTok{,}\StringTok{""}\NormalTok{),}\DataTypeTok{values =} \KeywordTok{c}\NormalTok{(}\StringTok{"lightblue"}\NormalTok{,}\StringTok{"lightsalmon"}\NormalTok{,}\StringTok{"gray"}\NormalTok{))}\OperatorTok{+}
\StringTok{    }\KeywordTok{geom_point}\NormalTok{(}\KeywordTok{aes}\NormalTok{(}\DataTypeTok{x =}\NormalTok{ x1, }\DataTypeTok{y=}\NormalTok{ x2, }\DataTypeTok{col =}\NormalTok{ y,}\DataTypeTok{shape=}\NormalTok{istest),}\DataTypeTok{size =}\DecValTok{1}\NormalTok{)}\OperatorTok{+}
\StringTok{    }\KeywordTok{scale_shape_manual}\NormalTok{(}\DataTypeTok{name =}\StringTok{""}\NormalTok{, }\DataTypeTok{label=}\KeywordTok{c}\NormalTok{(}\StringTok{"train"}\NormalTok{,}\StringTok{"test"}\NormalTok{),}\DataTypeTok{values=}\KeywordTok{c}\NormalTok{(}\DecValTok{3}\NormalTok{,}\DecValTok{16}\NormalTok{))}\OperatorTok{+}
\StringTok{    }\KeywordTok{guides}\NormalTok{(}\DataTypeTok{color=}\OtherTok{FALSE}\NormalTok{)}\OperatorTok{+}
\StringTok{    }\KeywordTok{geom_abline}\NormalTok{(}\KeywordTok{aes}\NormalTok{(}\DataTypeTok{slope=}\DecValTok{1}\NormalTok{,}\DataTypeTok{intercept=}\DecValTok{0}\NormalTok{,}\DataTypeTok{linetype =} \StringTok{"Decision boundary"}\NormalTok{))}\OperatorTok{+}
\StringTok{    }\KeywordTok{xlab}\NormalTok{(}\KeywordTok{expression}\NormalTok{(}\StringTok{"x"}\NormalTok{[}\DecValTok{1}\NormalTok{])) }\OperatorTok{+}
\StringTok{    }\KeywordTok{ylab}\NormalTok{(}\KeywordTok{expression}\NormalTok{(}\StringTok{"x"}\NormalTok{[}\DecValTok{2}\NormalTok{])) }\OperatorTok{+}
\StringTok{    }\KeywordTok{scale_x_continuous}\NormalTok{(}\DataTypeTok{limits =} \KeywordTok{c}\NormalTok{(}\OperatorTok{-}\DecValTok{100}\NormalTok{, }\DecValTok{20}\NormalTok{), }\DataTypeTok{expand =} \KeywordTok{c}\NormalTok{(}\DecValTok{0}\NormalTok{, }\DecValTok{0}\NormalTok{)) }\OperatorTok{+}
\StringTok{    }\KeywordTok{scale_y_continuous}\NormalTok{(}\DataTypeTok{limits =} \KeywordTok{c}\NormalTok{(}\OperatorTok{-}\DecValTok{100}\NormalTok{, }\DecValTok{20}\NormalTok{), }\DataTypeTok{expand =} \KeywordTok{c}\NormalTok{(}\DecValTok{0}\NormalTok{, }\DecValTok{0}\NormalTok{)) }\OperatorTok{+}
\StringTok{    }\KeywordTok{scale_linetype_manual}\NormalTok{(}\DataTypeTok{name=}\StringTok{"Argmax estimator"}\NormalTok{,}\DataTypeTok{labels=}\StringTok{"Decision Boundary"}\NormalTok{,}\DataTypeTok{values=}\StringTok{"dashed"}\NormalTok{)}
\end{Highlighting}
\end{Shaded}

\includegraphics{report_files/figure-latex/unnamed-chunk-13-1.pdf}

\begin{Shaded}
\begin{Highlighting}[]
\KeywordTok{print}\NormalTok{(}\StringTok{"Argmax predictor"}\NormalTok{)}
\NormalTok{cM_argmax <-}\StringTok{ }\KeywordTok{confusionMatrix}\NormalTok{(}\KeywordTok{table}\NormalTok{(quality_score}\OperatorTok{$}\NormalTok{y_hat[}\OperatorTok{-}\NormalTok{tr],quality_score}\OperatorTok{$}\NormalTok{y[}\OperatorTok{-}\NormalTok{tr]))}
\NormalTok{cM_argmax}
\KeywordTok{print}\NormalTok{(}\StringTok{"KNN predictor"}\NormalTok{)}
\NormalTok{cM_KNN <-}\StringTok{ }\KeywordTok{confusionMatrix}\NormalTok{(}\KeywordTok{table}\NormalTok{(KNNclass,quality_score}\OperatorTok{$}\NormalTok{y[}\OperatorTok{-}\NormalTok{tr]))}
\NormalTok{cM_KNN}
\end{Highlighting}
\end{Shaded}

\begin{verbatim}
## [1] "Argmax predictor"
## Confusion Matrix and Statistics
## 
##    
##       0   1
##   0 149  47
##   1  45 153
##                                           
##                Accuracy : 0.7665          
##                  95% CI : (0.7215, 0.8074)
##     No Information Rate : 0.5076          
##     P-Value [Acc > NIR] : <2e-16          
##                                           
##                   Kappa : 0.533           
##  Mcnemar's Test P-Value : 0.917           
##                                           
##             Sensitivity : 0.7680          
##             Specificity : 0.7650          
##          Pos Pred Value : 0.7602          
##          Neg Pred Value : 0.7727          
##              Prevalence : 0.4924          
##          Detection Rate : 0.3782          
##    Detection Prevalence : 0.4975          
##       Balanced Accuracy : 0.7665          
##                                           
##        'Positive' Class : 0               
##                                           
## [1] "KNN predictor"
## Confusion Matrix and Statistics
## 
##         
## KNNclass   0   1
##        0 139  45
##        1  55 155
##                                           
##                Accuracy : 0.7462          
##                  95% CI : (0.7002, 0.7884)
##     No Information Rate : 0.5076          
##     P-Value [Acc > NIR] : <2e-16          
##                                           
##                   Kappa : 0.4919          
##  Mcnemar's Test P-Value : 0.3681          
##                                           
##             Sensitivity : 0.7165          
##             Specificity : 0.7750          
##          Pos Pred Value : 0.7554          
##          Neg Pred Value : 0.7381          
##              Prevalence : 0.4924          
##          Detection Rate : 0.3528          
##    Detection Prevalence : 0.4670          
##       Balanced Accuracy : 0.7457          
##                                           
##        'Positive' Class : 0               
## 
\end{verbatim}

From from the confusion matrices we see that specificity is better for
the KNN predictor, but the sensitivity is better for the argmax
predictor. From the misclassficiation error for KNN, 0.2538071, and the
argmax, 0.2335025, We see that the argmax predictor is the best one.

\section{Problem 3: Bias-variance
trade-off}\label{problem-3-bias-variance-trade-off}

Here you see how to write formulas with latex (needed below) \[
\hat{\boldsymbol \beta}=({\bf X}^T{\bf X})^{-1}{\bf X}^T{\bf Y}
\]

\textbf{Q16:}

Utilizing that \(\text{E}(Y) = X\beta\) and
\(\text{E}(CX) = C\text{E}(X)\) we obtain that \(\hat{\beta}\) is an
unbiased estimator for \(\beta\).

\[
\text{E}(\hat{\beta}) = \text{E}\left[(X^TX)^{-1}X^T Y\right]  \\
= (X^TX)^{-1}X^T\text{E}(Y) = (X^TX)^{-1}X^TX\beta \\
= \beta
\] By using that \(\text{Cov}(CY) = C\text{Cov}(Y)C^T\) we obtain

\[
\text{Cov}(\hat{\beta}) = \text{Cov}\left[(X^TX)^{-1}X^TY\right] \\ 
= (X^TX)^{-1}X^T\text{Cov}(Y)\left[(X^TX)^{-1}X^T\right]^T \\
= \sigma^2(X^TX)^{-1}X^TX(X^TX)^{-1} \\ 
= \sigma^2(X^TX)^{-1}
\] \textbf{Q17:}

Using the same tools as above, we obtain

\[
\text{E}\left[\hat{f}(\bf{x}_0^T)\right] = \text{E}(\bf{x}_0^T\hat{\beta}) = \bf{x}_0^T\text{E}(\hat{\beta}) = x_0^T\beta
\] \[
\text{Var}\left[\hat{f}(\bf{x}_0^T)\right] = \text{Var}(\bf{x}_0^T\hat{\beta})
= \bf{x}_0^T\text{Cov}(\hat{\beta})\bf{x}_0 \\ 
= \sigma^2\bf{x}_0^T(X^TX)^{-1}\bf{x}_0
\]

\textbf{Q18:}
\[\text{E}[(Y_0-\hat{f}({\bf x}_0))^2]=[\text{E}(\hat{f}({\bf x}_0)-f({\bf x}_0)]^2+\text{Var}(\hat{f}({\bf x}_0) ) + \text{Var}(\varepsilon)\]

We start off by noting that
\(\text{E}(f(\bf{x}_0)) = \text{E}(Y_0 - \epsilon) = \text{E}(\bf{x}_0^T\beta)\),
which enables us to write \((\bf{x}_0^T - x_0^T\hat{\beta})^2\) as
\(\text{E}(f(\bf{x}_0) - \hat{f}(\bf{x}_0))^2\). The expression for
\(\text{E}\left[(Y_0 - \hat{f}(x_0))^2\right]\) becomes

\[
\text{E}\left[(Y_0 - \hat{f}(x_0))^2\right] =
\text{Var}(Y_0 - \hat{f}(x_0)) + \text{E}(Y_0 - \hat{f}(\bf{x}_0))^2 \\
= \text{Var}(\bf{x}_0^T\beta + \epsilon - \bf{x}_0^T\hat{\beta}) \\ + \left(\text{E}\left[\bf{x}_0^T\beta + \epsilon - \bf{x}_0^T\hat{\beta}\right]\right)^2 \\
= \text{Var}(\epsilon) + \text{Var}(\hat{f}(\bf{x}_0)) + \left(\text{E}\left[f(\bf{x}_0) - \hat{f}(\bf{x}_0)\right]\right)^2
\]

Ridge estimator: \[
\widetilde{\boldsymbol \beta}=({\bf X}^T{\bf X}+\lambda {\bf I})^{-1}{\bf X}^T{\bf Y}
\]

\textbf{Q19:}

We start off by defining \(G := X^TX\), and by rewriting \(\hat{\beta}\)
as \(\hat{\beta} = G^{-1}X^TY\). Note that G is a symmetric matrix. We
can now write \(\tilde{\beta}\) as

\[
\tilde{\beta} = (G - {\lambda}I)^{-1}X^TY = (G - {\lambda}I)^{-1}GG^{-1}X^TY = (G + {\lambda}I)^{-1}G\hat{\beta}
\] \[
= \left[G(I + {\lambda}G^{-1})\right]^{-1}G\hat{\beta} = (I + {\lambda}G^{-1})^{-1}G^{-1}G\hat{\beta} = (I + {\lambda}G^{-1})^{-1}\hat{\beta}
\] By defining \(H := (I + {\lambda}G^{-1})\), we end up with

\[
\tilde{\beta} = H^{-1}\hat{\beta}
\] It follows from the previous results, as well as the properties
introduced earlier that \[
\text{E}(\tilde{\beta}) = H^{-1}\text{E}(\hat{\beta}) = H^{-1}\beta
\] and

\[
\text{Cov}(\tilde{\beta}) = H^{-1}\text{Cov}(\hat{\beta})(H^{-1})^T = \sigma^2(HGH)^{-1}
\]

\textbf{Q20:}

\[
\text{E}(\bf{x}_0^T\tilde{\beta}) = \bf{x}_0^TH^{-1}\beta
\] \[
\text{Var}(\bf{x}_0^T\tilde{\beta}) = \sigma^2\bf{x}_0^T(HGH)^{-1}\bf{x}_0
\]

\textbf{Q21:}
\[\text{E}[(Y_0-\widetilde{f}({\bf x}_0))^2]=[\text{E}(\widetilde{f}({\bf x}_0)-f({\bf x}_0)]^2+\text{Var}(\widetilde{f}({\bf x}_0) ) + \text{Var}(\varepsilon)\]
Using that \(\text{Var}(X) = \text{E}(X^2) - \text{E}(X)^2\), we obtain

\[
\text{E}[(Y_0-\tilde{f}(\bf{x}_0))^2] = \text{Var}(\bf{x}_0^T\beta + \epsilon - \bf{x}_0^T\tilde{\beta}) + \left(\text{E}\left[\bf{x}_0^T\beta + \epsilon - \bf{x}_0^T\tilde{\beta}\right]\right)^2
\] \[
=\text{Var}(\epsilon) + \text{Var}(\tilde{f}(\bf{x_0})) + \left[\text{E}(f(\bf{x_0}) - \tilde{f}(\bf{x}_0) \right]^2
\]

\begin{Shaded}
\begin{Highlighting}[]
\NormalTok{values=}\KeywordTok{dget}\NormalTok{(}\StringTok{"https://www.math.ntnu.no/emner/TMA4268/2019v/data/BVtradeoffvalues.dd"}\NormalTok{)}
\NormalTok{X=values}\OperatorTok{$}\NormalTok{X}
\KeywordTok{dim}\NormalTok{(X)}
\NormalTok{x0=values}\OperatorTok{$}\NormalTok{x0}
\KeywordTok{dim}\NormalTok{(x0)}
\NormalTok{beta=values}\OperatorTok{$}\NormalTok{beta}
\KeywordTok{dim}\NormalTok{(beta)}
\NormalTok{sigma=values}\OperatorTok{$}\NormalTok{sigma}
\NormalTok{sigma}
\end{Highlighting}
\end{Shaded}

\begin{verbatim}
## [1] 100  81
## [1] 81  1
## [1] 81  1
## [1] 0.5
\end{verbatim}

Hint: we perform matrix multiplication using \texttt{\%*\%}, transpose
of a matrix \texttt{A} with \texttt{t(A)} and inverse with
\texttt{solve(A)}.

\textbf{Q22:}

The squared bias is zero for \(\lambda = 0\), as one would expect from
the results above. It was not expected to see that the squared bias is
approaching zero again around \(\lambda = 0.5\). Apart from this it
develops as one would expect by increasing as \(lambda\) increases.

\begin{Shaded}
\begin{Highlighting}[]
\NormalTok{sqbias=}\ControlFlowTok{function}\NormalTok{(lambda,X,x0,beta)}
\NormalTok{\{}
\NormalTok{  p=}\KeywordTok{dim}\NormalTok{(X)[}\DecValTok{2}\NormalTok{]}
\NormalTok{  G <-}\StringTok{ }\KeywordTok{t}\NormalTok{(X)}\OperatorTok\NormalTok{X}
\NormalTok{  H <-}\StringTok{ }\NormalTok{(}\KeywordTok{diag}\NormalTok{(p) }\OperatorTok{+}\StringTok{ }\NormalTok{lambda}\OperatorTok{*}\KeywordTok{solve}\NormalTok{(G))}
\NormalTok{  value <-}\StringTok{ }\NormalTok{(}\KeywordTok{t}\NormalTok{(x0)}\OperatorTok\NormalTok{beta }\OperatorTok{-}\StringTok{ }\KeywordTok{t}\NormalTok{(x0)}\OperatorTok\KeywordTok{solve}\NormalTok{(H)}\OperatorTok\NormalTok{beta)}\OperatorTok{^}\DecValTok{2} 
  \KeywordTok{return}\NormalTok{(value)}
\NormalTok{\}}

\NormalTok{thislambda=}\KeywordTok{seq}\NormalTok{(}\DecValTok{0}\NormalTok{,}\DecValTok{2}\NormalTok{,}\DataTypeTok{length=}\DecValTok{500}\NormalTok{)}
\NormalTok{sqbiaslambda=}\KeywordTok{rep}\NormalTok{(}\OtherTok{NA}\NormalTok{,}\KeywordTok{length}\NormalTok{(thislambda))}
\ControlFlowTok{for}\NormalTok{ (i }\ControlFlowTok{in} \DecValTok{1}\OperatorTok{:}\KeywordTok{length}\NormalTok{(thislambda)) sqbiaslambda[i]=}\KeywordTok{sqbias}\NormalTok{(thislambda[i],X,x0,beta)}
\KeywordTok{plot}\NormalTok{(thislambda,sqbiaslambda,}\DataTypeTok{col=}\DecValTok{2}\NormalTok{,}\DataTypeTok{type=}\StringTok{"l"}\NormalTok{)}
\end{Highlighting}
\end{Shaded}

\includegraphics{report_files/figure-latex/unnamed-chunk-16-1.pdf}

\textbf{Q23:}

Having seen the previous plot, and by remembering that the expected
squared prediction error is made up of the sum of the variance and the
squared bias we would expect to see that the variance drops as the
squared bias increases. Therefore, the plot of the variance is not
surprising.

\begin{Shaded}
\begin{Highlighting}[]
\NormalTok{variance=}\ControlFlowTok{function}\NormalTok{(lambda,X,x0,sigma)}
\NormalTok{\{}
\NormalTok{  p=}\KeywordTok{dim}\NormalTok{(X)[}\DecValTok{2}\NormalTok{]}
\NormalTok{  inv=}\KeywordTok{solve}\NormalTok{(}\KeywordTok{t}\NormalTok{(X)}\OperatorTok\NormalTok{X}\OperatorTok{+}\NormalTok{lambda}\OperatorTok{*}\KeywordTok{diag}\NormalTok{(p))}
\NormalTok{  G <-}\StringTok{ }\KeywordTok{t}\NormalTok{(X)}\OperatorTok\NormalTok{X}
\NormalTok{  H <-}\StringTok{ }\NormalTok{(}\KeywordTok{diag}\NormalTok{(p) }\OperatorTok{+}\StringTok{ }\NormalTok{lambda}\OperatorTok{*}\KeywordTok{solve}\NormalTok{(G))}
\NormalTok{  value <-}\StringTok{ }\NormalTok{sigma}\OperatorTok{^}\DecValTok{2}\OperatorTok{*}\KeywordTok{t}\NormalTok{(x0)}\OperatorTok\KeywordTok{solve}\NormalTok{(H}\OperatorTok\NormalTok{G}\OperatorTok\NormalTok{H)}\OperatorTok\NormalTok{x0}
  \KeywordTok{return}\NormalTok{(value)}
\NormalTok{\}}
\NormalTok{thislambda=}\KeywordTok{seq}\NormalTok{(}\DecValTok{0}\NormalTok{,}\DecValTok{2}\NormalTok{,}\DataTypeTok{length=}\DecValTok{500}\NormalTok{)}
\NormalTok{variancelambda=}\KeywordTok{rep}\NormalTok{(}\OtherTok{NA}\NormalTok{,}\KeywordTok{length}\NormalTok{(thislambda))}
\ControlFlowTok{for}\NormalTok{ (i }\ControlFlowTok{in} \DecValTok{1}\OperatorTok{:}\KeywordTok{length}\NormalTok{(thislambda)) variancelambda[i]=}\KeywordTok{variance}\NormalTok{(thislambda[i],X,x0,sigma)}
\KeywordTok{plot}\NormalTok{(thislambda,variancelambda,}\DataTypeTok{col=}\DecValTok{4}\NormalTok{,}\DataTypeTok{type=}\StringTok{"l"}\NormalTok{)}
\end{Highlighting}
\end{Shaded}

\includegraphics{report_files/figure-latex/unnamed-chunk-17-1.pdf}

\textbf{Q24:}

\begin{Shaded}
\begin{Highlighting}[]
\NormalTok{tot=sqbiaslambda}\OperatorTok{+}\NormalTok{variancelambda}\OperatorTok{+}\NormalTok{sigma}\OperatorTok{^}\DecValTok{2}
\KeywordTok{which.min}\NormalTok{(tot)}
\NormalTok{thislambda[}\KeywordTok{which.min}\NormalTok{(tot)]}
\KeywordTok{plot}\NormalTok{(thislambda,tot,}\DataTypeTok{col=}\DecValTok{1}\NormalTok{,}\DataTypeTok{type=}\StringTok{"l"}\NormalTok{,}\DataTypeTok{ylim=}\KeywordTok{c}\NormalTok{(}\DecValTok{0}\NormalTok{,}\KeywordTok{max}\NormalTok{(tot)))}
\KeywordTok{lines}\NormalTok{(thislambda, sqbiaslambda,}\DataTypeTok{col=}\DecValTok{2}\NormalTok{)}
\KeywordTok{lines}\NormalTok{(thislambda, variancelambda,}\DataTypeTok{col=}\DecValTok{4}\NormalTok{)}
\KeywordTok{lines}\NormalTok{(thislambda,}\KeywordTok{rep}\NormalTok{(sigma}\OperatorTok{^}\DecValTok{2}\NormalTok{,}\DecValTok{500}\NormalTok{),}\DataTypeTok{col=}\StringTok{"orange"}\NormalTok{)}
\KeywordTok{abline}\NormalTok{(}\DataTypeTok{v=}\NormalTok{thislambda[}\KeywordTok{which.min}\NormalTok{(tot)],}\DataTypeTok{col=}\DecValTok{3}\NormalTok{)}
\end{Highlighting}
\end{Shaded}

\includegraphics{report_files/figure-latex/unnamed-chunk-18-1.pdf}

\begin{verbatim}
## [1] 249
## [1] 0.993988
\end{verbatim}

The optimal value for \(\lambda\) is shown below. An alternative
approach to plotting and finding this optimal value is also presented
below. It serves the same purpose as what is done above.

\begin{Shaded}
\begin{Highlighting}[]
\KeywordTok{library}\NormalTok{(tidyverse)}
\end{Highlighting}
\end{Shaded}

\begin{verbatim}
## -- Attaching packages ------------------------------------------------------------------------------- tidyverse 1.2.1 --
\end{verbatim}

\begin{verbatim}
## v tibble  2.0.1       v purrr   0.3.0  
## v tidyr   0.8.2       v dplyr   0.8.0.1
## v readr   1.3.1       v stringr 1.4.0  
## v tibble  2.0.1       v forcats 0.4.0
\end{verbatim}

\begin{verbatim}
## -- Conflicts ---------------------------------------------------------------------------------- tidyverse_conflicts() --
## x dplyr::filter() masks stats::filter()
## x dplyr::lag()    masks stats::lag()
## x purrr::lift()   masks caret::lift()
\end{verbatim}

\begin{Shaded}
\begin{Highlighting}[]
\NormalTok{lambdas <-}\StringTok{ }\KeywordTok{seq}\NormalTok{(}\DecValTok{0}\NormalTok{, }\DecValTok{2}\NormalTok{, }\DataTypeTok{length =} \DecValTok{500}\NormalTok{)}
\NormalTok{df <-}\StringTok{ }\KeywordTok{tibble}\NormalTok{(}\DataTypeTok{lambda =}\NormalTok{ lambdas,}
             \DataTypeTok{sqbias =}\NormalTok{ lambdas }\OperatorTok
\StringTok{               }\KeywordTok{map_dbl}\NormalTok{(}\KeywordTok{partial}\NormalTok{(sqbias, }\DataTypeTok{x0 =}\NormalTok{ x0, }\DataTypeTok{beta =}\NormalTok{ beta, }\DataTypeTok{X =}\NormalTok{ X)),}
             \DataTypeTok{variance =}\NormalTok{ lambdas }\OperatorTok
\StringTok{               }\KeywordTok{map_dbl}\NormalTok{(}\KeywordTok{partial}\NormalTok{(variance, }\DataTypeTok{x0 =}\NormalTok{ x0, }\DataTypeTok{sigma =}\NormalTok{ sigma, }\DataTypeTok{X =}\NormalTok{ X)))}
\NormalTok{df <-}\StringTok{ }\NormalTok{df }\OperatorTok\StringTok{ }\KeywordTok{mutate}\NormalTok{(}\DataTypeTok{tot =}\NormalTok{ sqbias }\OperatorTok{+}\StringTok{ }\NormalTok{variance }\OperatorTok{+}\StringTok{ }\NormalTok{sigma}\OperatorTok{^}\DecValTok{2}\NormalTok{)}
\NormalTok{min <-}\StringTok{ }\NormalTok{df }\OperatorTok\StringTok{ }\KeywordTok{filter}\NormalTok{(tot }\OperatorTok{==}\StringTok{ }\KeywordTok{min}\NormalTok{(tot)) }\OperatorTok\StringTok{ }\KeywordTok{select}\NormalTok{(lambda, tot)}
\NormalTok{df }\OperatorTok\StringTok{ }\KeywordTok{ggplot}\NormalTok{(}\KeywordTok{aes}\NormalTok{(}\DataTypeTok{x =}\NormalTok{ lambda)) }\OperatorTok{+}
\StringTok{  }\KeywordTok{geom_line}\NormalTok{(}\KeywordTok{aes}\NormalTok{(}\DataTypeTok{y =}\NormalTok{ variance, }\DataTypeTok{color =} \StringTok{'Variance'}\NormalTok{)) }\OperatorTok{+}
\StringTok{  }\KeywordTok{geom_line}\NormalTok{(}\KeywordTok{aes}\NormalTok{(}\DataTypeTok{y =}\NormalTok{ sqbias, }\DataTypeTok{color =} \StringTok{'Squared bias'}\NormalTok{)) }\OperatorTok{+}
\StringTok{  }\KeywordTok{geom_line}\NormalTok{(}\KeywordTok{aes}\NormalTok{(}\DataTypeTok{y =}\NormalTok{ tot, }\DataTypeTok{color =} \StringTok{'Expected squared error'}\NormalTok{)) }\OperatorTok{+}
\StringTok{  }\KeywordTok{geom_hline}\NormalTok{(}\KeywordTok{aes}\NormalTok{(}\DataTypeTok{color =} \StringTok{'Irreducible error'}\NormalTok{, }\DataTypeTok{yintercept =}\NormalTok{ sigma}\OperatorTok{^}\DecValTok{2}\NormalTok{)) }\OperatorTok{+}\StringTok{ }
\StringTok{  }\KeywordTok{geom_vline}\NormalTok{(}\KeywordTok{aes}\NormalTok{(}\DataTypeTok{xintercept =}\NormalTok{ min }\OperatorTok\StringTok{ }\KeywordTok{select}\NormalTok{(lambda) }\OperatorTok\StringTok{ }\KeywordTok{as.numeric}\NormalTok{())) }\OperatorTok{+}
\StringTok{  }\KeywordTok{xlab}\NormalTok{(}\KeywordTok{expression}\NormalTok{(lambda)) }\OperatorTok{+}\StringTok{ }
\StringTok{  }\KeywordTok{ylab}\NormalTok{(}\OtherTok{NULL}\NormalTok{) }\OperatorTok{+}\StringTok{ }
\StringTok{  }\KeywordTok{ggtitle}\NormalTok{(}\StringTok{'Components of expected squared error'}\NormalTok{) }\OperatorTok{+}
\StringTok{  }\KeywordTok{theme}\NormalTok{(}\DataTypeTok{legend.title =} \KeywordTok{element_blank}\NormalTok{())}
\end{Highlighting}
\end{Shaded}

\includegraphics{report_files/figure-latex/unnamed-chunk-19-1.pdf}

\begin{Shaded}
\begin{Highlighting}[]
\NormalTok{min }\OperatorTok\StringTok{ }\NormalTok{knitr}\OperatorTok{::}\KeywordTok{kable}\NormalTok{(}\DataTypeTok{caption =} \StringTok{'Optimal expected squared error'}\NormalTok{,}
              \DataTypeTok{col.names =} \KeywordTok{c}\NormalTok{(}\StringTok{'$}\CharTok{\textbackslash{}\textbackslash{}}\StringTok{lambda$'}\NormalTok{, }\StringTok{'Expected squared error'}\NormalTok{),}
              \DataTypeTok{escape=}\OtherTok{FALSE}\NormalTok{)}
\end{Highlighting}
\end{Shaded}

\begin{longtable}[]{@{}rr@{}}
\caption{Optimal expected squared error}\tabularnewline
\toprule
\(\lambda\) & Expected squared error\tabularnewline
\midrule
\endfirsthead
\toprule
\(\lambda\) & Expected squared error\tabularnewline
\midrule
\endhead
0.993988 & 0.4870826\tabularnewline
\bottomrule
\end{longtable}


\end{document}
